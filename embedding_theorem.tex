\documentclass[11pt]{article}

\usepackage{amsmath, amsthm, amssymb}
\usepackage{enumerate}
\usepackage{pdflscape}
\usepackage{caption}
\usepackage{bm}

\usepackage{ifpdf}
\ifpdf
\usepackage[pdftex]{graphicx}
\else
\usepackage[dvips]{graphicx}
\fi
\usepackage{tikz}
 	 \usetikzlibrary{arrows,backgrounds}
\usepackage[all�]{xy}

\usepackage{multicol}

\usepackage{tocvsec2}

\usepackage{bbm}

\input xy
\xyoption{all}

\usepackage[pdftex,plainpages=false,hypertexnames=false,pdfpagelabels]{hyperref}
\newcommand{\arxiv}[1]{\href{http://arxiv.org/abs/#1}{\tt arXiv:\nolinkurl{#1}}}
\newcommand{\arXiv}[1]{\href{http://arxiv.org/abs/#1}{\tt arXiv:\nolinkurl{#1}}}
\newcommand{\doi}[1]{\href{http://dx.doi.org/#1}{{\tt DOI:#1}}}
\newcommand{\euclid}[1]{\href{http://projecteuclid.org/getRecord?id=#1}{{\tt #1}}}
\newcommand{\mathscinet}[1]{\href{http://www.ams.org/mathscinet-getitem?mr=#1}{\tt #1}}
\newcommand{\googlebooks}[1]{(preview at \href{http://books.google.com/books?id=#1}{google books})}
\newcommand{\Ws}{\text{W*}}

\usepackage{xcolor}
\definecolor{dark-red}{rgb}{0.7,0.25,0.25}
\definecolor{dark-blue}{rgb}{0.15,0.15,0.55}
\definecolor{medium-blue}{rgb}{0,0,.8}
\definecolor{shaded-blue}{RGB}{98,140,255}
\definecolor{DarkGreen}{RGB}{0,150,0}
\definecolor{rho}{named}{red}
\definecolor{Salmon}{RGB}{255, 144, 144}
\hypersetup{
   colorlinks, linkcolor={purple},
   citecolor={medium-blue}, urlcolor={medium-blue}
}

%\addtolength{\textwidth}{.5in}
\usepackage{longtable}
\usepackage{fullpage}
%\renewcommand{\arraystretch}{1.5}

% Page size %%%%%%%%%%%%%%%%%%%%%%%%%%%%%%%%%%%%%%%%%%%
\setlength\topmargin{-.25in}
\setlength\headheight{0in}
\setlength\headsep{.2in}
\setlength\textheight{9in}
%\addtolength{\hoffset}{-0.25in}
%\addtolength{\textwidth}{.5in}
\setlength\parindent{0.25in}

% Theorems %%%%%%%%%%%%%%%%%%%%%%%%%%%%%%%%%%%%%%%%%%
\theoremstyle{plain}
\newtheorem{thm}{Theorem}[section]
\newtheorem*{thm*}{Theorem}
\newtheorem{thmalpha}{Theorem}
\renewcommand*{\thethmalpha}{\Alph{thmalpha}}
\newtheorem{cor}[thm]{Corollary}
\newtheorem{coralpha}[thmalpha]{Corollary}
\newtheorem*{cor*}{Corollary}
\newtheorem{conj}[thm]{Conjecture}
\newtheorem{conjalpha}[thmalpha]{Conjecture}
\newtheorem*{conj*}{Conjecture}
\newtheorem{lem}[thm]{Lemma}
\newtheorem{fact}[thm]{Fact}
\newtheorem{facts}[thm]{Facts}
\newtheorem{prop}[thm]{Proposition}
\newtheorem{quest}[thm]{Question}
\newtheorem*{quest*}{Question}
\newtheorem*{claim*}{Claim}
\newtheorem{quests}[thm]{Questions}
\theoremstyle{definition}
\newtheorem{defn}[thm]{Definition}
\newtheorem{construction}[thm]{Construction}
\newtheorem{alg}[thm]{Algorithm}
\newtheorem{assumption}[thm]{Assumption}
\newtheorem{nota}[thm]{Notation}
\newtheorem{nb}[thm]{Note}
\newtheorem{note}[thm]{Note}
\newtheorem{exs}[thm]{Examples}
\newtheorem{ex}[thm]{Example}
\newtheorem{exercise}[thm]{Exercise}
\newtheorem{sub-ex}[thm]{Sub-Example}
\newtheorem{rem}[thm]{Remark}
\newtheorem*{rem*}{Remark}
\newtheorem{remark}[thm]{Remark}
\newtheorem{rems}[thm]{Remarks}
\newtheorem{warn}[thm]{Warning}

% Figure Numbering %%%%%%%%%%%%%%%%%%%%%%%%%%%%%%%%%%%
%\usepackage{chngcntr} %Numbers figures by section
%\counterwithin{figure}{section}

% Operators %%%%%%%%%%%%%%%%%%%%%%%%%%%%%%%%%%%%%%%%%%%
\DeclareMathOperator{\Ad}{Ad}
\DeclareMathOperator{\Aut}{Aut}
\DeclareMathOperator{\coev}{coev}
\DeclareMathOperator{\Dom}{Dom}
\DeclareMathOperator{\End}{End}
\DeclareMathOperator{\ev}{ev}
\DeclareMathOperator{\Hom}{Hom}
\DeclareMathOperator{\Mor}{Mor}
\DeclareMathOperator{\op}{op}
\DeclareMathOperator{\ONB}{ONB}
\DeclareMathOperator{\Ob}{Ob}
\DeclareMathOperator{\rev}{rev}
\DeclareMathOperator{\spann}{span}
\DeclareMathOperator{\supp}{supp}
\DeclareMathOperator{\id}{id}
\DeclareMathOperator{\Isom}{Isom}
\DeclareMathOperator{\ind}{ind}
\DeclareMathOperator{\im}{im}
\DeclareMathOperator{\Irr}{Irr}
\DeclareMathOperator{\Spec}{Spec}
\DeclareMathOperator{\Stab}{Stab}
\DeclareMathOperator{\Tr}{Tr}
\DeclareMathOperator{\tr}{tr}
\DeclareMathOperator{\Gr}{Gr}


% Math %%%%%%%%%%%%%%%%%%%%%%%%%%%%%%%%%%%%%%%%%%%%%
\newcommand{\D}{\displaystyle}
\newcommand{\comment}[1]{}
\newcommand{\hs}{\hspace{.07in}}
\newcommand{\hsp}[1]{\hs\text{#1}\hs}
\newcommand{\be}{\begin{enumerate}[label=(\arabic*)]}
\newcommand{\ee}{\end{enumerate}}
\newcommand{\itm}[1]{\item[\underline{\ensuremath{#1}:}]}
\newcommand{\itt}[1]{\item[\underline{\text{#1}:}]}
\newcommand{\N}{\mathbb{N}}
\newcommand{\Natural}{\mathbb{N}}
\newcommand{\Z}{\mathbb{Z}}
\newcommand{\Q}{\mathbb{Q}}
\newcommand{\F}{\mathbb{F}}
\newcommand{\R}{\mathbb{R}}
\newcommand{\C}{\mathbb{C}}
\newcommand{\B}{\mathbb{B}}
\renewcommand{\P}{\mathbb{P}}
\newcommand{\I}{\infty}
\newcommand{\set}[2]{\left\{#1 \middle| #2\right\}}
\newcommand{\thh}{^{\text{th}}}
\renewcommand{\a}{\mathfrak{a}}
\renewcommand{\b}{\mathfrak{b}}
\renewcommand{\c}{\mathfrak{c}}
\newcommand{\n}{\mathfrak{n}}
\newcommand{\m}{\mathfrak{m}}
\newcommand{\bbOne}{\mathbbm{1}}
\renewcommand{\alg}[1]{{\bm{\langle} #1\bm{\rangle}}}


% some math commands specific to this document %%%%%%%%%%%%
\newcommand{\xalt}{x^{\text{alt}\otimes n}}
\newcommand{\xbaralt}{\overline{x}^{\text{alt}\otimes n}}
\newcommand{\act}{\overline{x}^{?}}
%%%%%%%%%%%%%%%%%%%%%%%%%%%%

\newcommand{\dave}[1]{\marginpar{\tiny \textcolor{orange}{DP: #1}}}
\newcommand{\corey}[1]{\marginpar{\tiny \textcolor{green}{CJ: #1}}}
\newcommand{\Af}{\mathcal{A}\Lambda_{F}}
\newcommand{\WStar}{\bfW\text{*}}
%\newcommand{\Irr}{\text{Irr}(\mathcal{C})}

% tricky way to iterate macros over a list
\def\semicolon{;}
\def\applytolist#1{
    \expandafter\def\csname multi#1\endcsname##1{
        \def\multiack{##1}\ifx\multiack\semicolon
            \def\next{\relax}
        \else
            \csname #1\endcsname{##1}
            \def\next{\csname multi#1\endcsname}
        \fi
        \next}
    \csname multi#1\endcsname}

% \def\cA{{\cal A}} for A..Z
\def\calc#1{\expandafter\def\csname c#1\endcsname{{\mathcal #1}}}
\applytolist{calc}QWERTYUIOPLKJHGFDSAZXCVBNM;
% \def\bbA{{\mathbb A}} for A..Z
\def\bbc#1{\expandafter\def\csname bb#1\endcsname{{\mathbb #1}}}
\applytolist{bbc}QWERTYUIOPLKJHGFDSAZXCVBNM;
% \def\bfA{{\mathbf A}} for A..Z
\def\bfc#1{\expandafter\def\csname bf#1\endcsname{{\mathbf #1}}}
\applytolist{bfc}QWERTYUIOPLKJHGFDSAZXCVBNM;
% \def\sA{{\sf A}} for A..Z
\def\sfc#1{\expandafter\def\csname s#1\endcsname{{\sf #1}}}
\applytolist{sfc}QWERTYUIOPLKJHGFDSAZXCVBNM;
% \def\fA{{\mathfrak A}} for A..Z
\def\fc#1{\expandafter\def\csname f#1\endcsname{{\mathfrak #1}}}
\applytolist{fc}QWERTYUIOPLKJHGFDSAZXCVBNM;


\newcommand{\PA}{\cP\hspace{-.1cm}\cA}
\newcommand{\Fun}{{\sf Fun}}
\newcommand{\Rep}{{\sf Rep}}
\newcommand{\Set}{{\sf Set}}
\newcommand{\FreeMod}{{\sf FreeMod}}
\newcommand{\Mod}{{\sf Mod}}
\newcommand{\Proj}{{\sf Proj}}
\newcommand{\AlgBim}{{\sf AlgBim}}
\newcommand{\Bim}{{\sf Bim}}
\newcommand{\bfBim}{{\sf Bim_{bf}}}
\newcommand{\spBim}{{\sf Bim^{sp}}}
\newcommand{\spbfBim}{{\sf Bim_{bf}^{sp}}}
\renewcommand{\Vec}{{\sf Vec}}
\newcommand{\fdVec}{{\sf Vec_{fd}}}
\newcommand{\Hilb}{{\sf Hilb}}
\newcommand{\fdHilb}{{\sf Hilb_{fd}}}
\newcommand{\ConAlg}{{\sf ConAlg}}
\newcommand{\DisInc}{{\sf DisInc}}


\newcommand{\jw}[1]{f^{(#1)}}
\newcommand{\todo}[1]{\textcolor{blue}{\textbf{TODO: #1}}}
\newcommand{\nn}[1]{\textcolor{red}{[[#1]]}}
\newcommand{\noshow}[1]{}
\newcommand{\MR}[1]{}
\newcommand{\TL}{\cT\hspace{-.08cm}\cL}
\newcommand{\rhoE}{\textcolor{rho}{e_1}}
\newcommand{\rhoJW}{\textcolor{rho}{\jw{2}}}


% TikZ operators %%%%%%%%%%%%%%%%%%%%%%%%%%%%%%%%%%%%%%%%
\usetikzlibrary{shapes}
\usetikzlibrary{cd}
\usetikzlibrary{backgrounds}
\usetikzlibrary{decorations,decorations.pathreplacing,decorations.markings}
\usetikzlibrary{fit,calc,through}
\usetikzlibrary{external}
\usetikzlibrary{arrows}
\tikzset{vertex/.style = {shape=circle,draw,fill=black,inner sep=0pt,minimum size=5pt}}
\tikzset{edge/.style = {->,> = latex', bend right}}
\tikzset{
	super thick/.style={line width=3pt}
}
\tikzset{
    quadruple/.style args={[#1] in [#2] in [#3] in [#4]}{
        #1,preaction={preaction={preaction={draw,#4},draw,#3}, draw,#2}
    }
}
\tikzstyle{shaded}=[fill=gray!25!white]
\tikzstyle{shadedpink}=[left color= white, right color = Salmon]
\tikzstyle{unshaded}=[fill=white]
\tikzstyle{empty box}=[circle, draw, thick, fill=white, opaque, inner sep=2mm]
\tikzstyle{annular}=[scale=.7, inner sep=1mm, baseline]
\tikzstyle{rectangular}=[scale=.75, inner sep=1mm, baseline=-.1cm]
\tikzstyle{mid>}=[decoration={markings, mark=at position 0.5 with {\arrow{>}}}, postaction={decorate}]
\tikzstyle{mid<}=[decoration={markings, mark=at position 0.5 with {\arrow{<}}}, postaction={decorate}]
\tikzstyle{over}=[double, draw=white, super thick, double=]
\tikzstyle{relativecommutantshading}=[fill=blue!40!white]
\tikzstyle{ctwoshading}=[fill=red!15!white]

\newcommand{\roundNbox}[6]{
	\draw[rounded corners=5pt, very thick, #1] ($#2+(-#3,-#3)+(-#4,0)$) rectangle ($#2+(#3,#3)+(#5,0)$);
	\coordinate (ZZa) at ($#2+(-#4,0)$);
	\coordinate (ZZb) at ($#2+(#5,0)$);
	\node at ($1/2*(ZZa)+1/2*(ZZb)$) {#6};
}


\newcommand{\nbox}[6]{
	\draw[thick, #1] ($#2+(-#3,-#3)+(-#4,0)$) rectangle ($#2+(#3,#3)+(#5,0)$);
	\coordinate (ZZa) at ($#2+(-#4,0)$);
	\coordinate (ZZb) at ($#2+(#5,0)$);
	\node at ($1/2*(ZZa)+1/2*(ZZb)$) {#6};
}

\newcommand{\ncircle}[4]{
	\draw[very thick, #1] #2 circle (#3);
	\node at #2 {#4};
%	\filldraw[red] ($#2+(#4:#3cm)$) circle (.05cm);
%	\node at ($#2+(#4:.15cm)+(#4:#3cm)$) {$\star$};
}

  \newcommand{\tikzmath}[2][]
     {\vcenter{\hbox{\begin{tikzpicture}[#1]#2
                     \end{tikzpicture}}}
     }

% premade TIKZ stuff %%%%%%%%%%%%%%%%%%%%%%%%%%%%
\newcommand{\uscircle}{
\begin{tikzpicture}
\draw[thick] (0,0) circle (1mm) ;
\end{tikzpicture}
}

\newcommand{\scircle}{
\begin{tikzpicture}
\filldraw[thick][shaded] (0,0) circle (1mm);
\end{tikzpicture}
}

% colors %%%%%%%%%%%%%%%%%%%%%%%%%%%%%%%%%%%%%%%%
\newcommand{\cupcolor}{DarkGreen}
\newcommand{\alphacolor}{blue}
\newcommand{\betacolor}{red}

% title %%%%%%%%%%%%%%%%%%%%%%%%%%%%%%%%%%%%%%%%%

\title{Module embedding via towers of algebras}
\author{Desmond Coles, Peter Huston, David Penneys, \& Srivatsa Srinivas}





%%%%%%%%%%%%%%%%%%%%%%%%%%%%%%%%%%%%%%


\usepackage[utf8]{inputenc}
\begin{document}


%%%%%%%%%%%%%%%%%%%%%%%%%%%%%%%%%%%%%%%%%%%%%%%%%%%%%%%%%%%%%%%
%%%%%%%%%%%%%%%%%%%%%%%%%%%%%%%%%%%%%%%%%%%%%%%%%%%%%%%%%%%%%%%
%%%%%%%%%%%%%%%%%%%%%%%%%%%%%%%%%%%%%%%%%%%%%%%%%%

\maketitle
\begin{abstract}
Jones and Penneys have shown that a finite depth subfactor planar $*$-algebra embeds in the bipartite graph planar algebra of its principal graph. By constructing a strongly Markov inclusion of finite von Neumann algebras from a given module, we extend their techniques to the case of cyclic modules over a subfactor planar algebra, relating the calculus of string diagrams in a module category and the canonical planar $*$-algebra structure on a Markov inclusion. We generalize their result, showing that a finite depth subfactor $*$-planar algebra embeds into the bipartite graph planar algebra of the fusion graph of any of its cyclic modules. 
\end{abstract}
\section{Introduction}
\section{Preliminaries}
\subsection{Planar Algebras and the Paragroup}  \label{cattheory}
Planar algebras were originally introduced in \cite{planaralg1}. We will use the same definitions and conventions as \cite{penneys}, \cite{peters}, \cite{jones}, and \cite{planaralg2}. The reader is referred to these references for further information on planar algebras. From any subfactor planar algebra, one may construct a 2-category. We refer the reader to \cite{paragroup} and \cite{paragroup2} for more details. We adopt the definitions used in \cite{paragroup}, since we wish to work with shaded planar algebras.

\begin{defn}\textnormal{
		Let $Q_{\bullet}$ be a subfactor planar algebra. The \textit{paragroup} of $Q_\bullet$ is a 2-category $\cG$: 
\begin{itemize}
\item The 0-morphisms of $\cG$ are the two objects
$\uscircle$ and $\scircle$.
\item The 1-morphisms are projections in the box spaces of $Q_{\bullet}$, as follows:
\begin{itemize}
	\item $\Hom_{\cG}(\uscircle\rightarrow\uscircle)=\{p\in Q_{i,+}|\text{$p$ is projection and $i$ is even}\}$
	\item $\Hom_{\cG}(\uscircle\rightarrow\scircle)=\{p\in Q_{i,+}|\text{$p$ is projection and $i$ is odd}\}$
	\item $\Hom_{\cG}(\scircle\rightarrow\uscircle)=\{p\in Q_{i,-}|\text{$p$ is projection and $i$ is odd}\}$
	\item $\Hom_{\cG}(\scircle\rightarrow\scircle)=\{p\in Q_{i,-}|\text{$p$ is projection and $i$ is even}\}$
\end{itemize}
Composition of $1$-morphisms is denoted by $\otimes$ and represented diagrammatically by horizontal concatenation of diagrams.
\item The 2-morphisms of $\cG$ are defined as follows: for $p\in P_{i,\pm}$ and $q\in P_{i,\pm}$, we set $\Hom_{\cG}(p\rightarrow q)=qP_{j\rightarrow i}p$. The space $P_{j\rightarrow i}$ is just $P_{i+j}$, except that tangles are depicted with $j$ with strings emanating from the top and $i$ from the bottom. Note that by construction, $i+j$ is always even. 
\end{itemize}}
\end{defn}

For a general subfactor planar algebra $Q_{\bullet}$, the $2$-category $\cG$ can be interpreted as a unitary multitensor category $\cC$ with the canonical spherical unitary dual functor following \cite{egno,UnitaryDualFunctors}. 
When $Q_{\bullet}$ is finite depth, $\cC$ is a $2\times 2$ unitary multifusion category. 

By a module over $Q_\bullet$, we mean a $\ast$-module category over the $\ast$-multitensor category $\cC$. Let $\cM$ be a cyclic left $\cC$-module category with generator $m$. 
Since $\cM$ is indecomposable, we can take $m$ to be simple. 
By \cite{ostrik03}, there is a $\ast$-algebra $A$ internal to $\cC$ such that $\cM\cong\FreeMod_\cC(A)$, the category of free right $A$-modules internal to $\cC$, namely the $\cC$-valued internal $\End$ of $m$. 
Since $m$ is simple, 
$A$ is in fact a Q-system, % Q-system or $Q$-system?
as described in \cite[Rmk.2.7]{1704.04729}.  
Each module in this category is of the form $X\otimes A$ for some $X\in\cC$. 
Each $X$ in $\cC$ is trivially a bimodule over the tensor unit $1_{0,+}\oplus1_{0,-}$, so each $X\in\Irr(\cC)$ is a module over either $1_{0,+}$ or $1_{0,-}$ on each side. 
We can therefore view $\cG$ as the $2$-category of bimodules over the algebra objects $1_{0,+}$ and $1_{0,-}$ and view $\cM$ as the category of $1_{0,+}-A$ and $1_{0,-}-A$ bimodules. 
By \cite[Thm.4.1]{1704.04729}, we can again obtain a $3\times 3$ unitary multifusion category $\widetilde{\cC}$ from the $2$-category of bimodules over the algebras $1_{0,+}$, $1_{0,-}$, and $A$, into which both $\cC$ and $\cM$ have naturally monoidal embeddings. 
Taking the full subcategory of objects which are preserved by tensoring by the appropriate components of the tensor unit allows us to recover a $2\time 2$ multifusion subcategory $\cC$ and a cyclic $\cC$-module $\cM$. 



\subsection{Markov Sequences}


\begin{defn}
A \emph{Markov sequence} consists of a sequence $(M_n, \tr_n)_{n\geq 0}$ of finite dimensional von Neumann algebras, such that $M_n$ is unitally included in $M_{n_1}$, each $M_n$ has a faithful normal tracial states such that $\tr_{n+1}|_{M_n} = \tr_n$ for all $n\geq 0$, and there is a sequence of \emph{Jones projections} $e_n \in M_{n+1}$ for all $n\geq 1$, such that:
\begin{itemize}
\item
The projections $(e_n)$ satisfy the Temperley-Lieb-Jones relations:
\begin{enumerate}[(1)]
\item
$e_i^2 = e_i = e_i^*$ for all $i$,
\item
$e_i e_j = e_j e_i$ for $|i-j|>1$, and
\item
there is a fixed constant $d>0$ such that $e_{i} e_{i\pm 1} e_i = d^{-2} e_i$ for all $i$.
\end{enumerate}
\item
For all $x\in M_n$, $e_n x e_n = E_n(x)e_n$, where $E_n: M_n \to M_{n-1}$ is the canonical faithful trace-preserving conditional expectation.
\item
For all $n\geq 1$, $E_{n+1}(e_n) = d^{-2}$.
\item
(pull down)
For all $n\geq 1$, $M_{n+1}e_n = M_n e_n$.

\end{itemize}
\end{defn}


\begin{rem}\label{pulldowniff}
$M_n e_n M_n$ is a 2-sided ideal in $M_{n+1}$ for all $n\geq 1$ if and only if the pull down condition holds. Indeed, if the pull down condition holds, then $M_{n+1} M_n e_n M_n \subseteq M_{n+1} e_n M_n = M_n e_n M_n$; the same argument holds on the right by first taking adjoints. Conversely, if $M_n e_n M_n$ is a 2-sided ideal, then $M_{n+1} e_n = (M_{n+1} e_n)e_n \subseteq (M_n e_n M_n) e_n = M_n e_n$.
\end{rem}




\begin{prop}\label{prop:ElementaryMarkov} A Markov sequence satisfies the following elementary properties for $n\geq 1$.
\begin{enumerate}[(A)]
\item
\label{EP:Injective}
The map $M_{n}\ni y\mapsto ye_n \in M_{n+1}$ is injective.

\item
\label{EP:UniquePullDown}
For all $x\in M_{n+1}$, $d^{2}E_{n+1}(x e_n)$ is the unique element $y\in M_n$ such that $x e_n = ye_n$ \cite[Lem.~1.2]{MR860811}.

\item
\label{EP:MarkovTraces}
The traces $\tr_{n+1}$ satisfy the following \emph{Markov property} with respect to $M_n$ and $e_n$: for all $x\in M_n$, $\tr_{n+1}(xe_n) = d^{-2} \tr_n(x)$.

\item
\label{EP:CompressM_{n+1}}
$e_n M_{n+1}e_n = M_{n-1}e_n$.

\item
\label{EP:2SidedIdeal}
$X_{n+1}:=M_n e_n M_n$ is a 2-sided ideal of $M_{n+1}$, and thus $M_{n+1}$ splits as a direct sum of von Neumann algebras $X_{n+1}\oplus Y_{n+1}$.
(In \cite[Thm.~4.1.4 and Thm.~4.6.3]{MR999799}, $Y_{n+1}$ is the so-called `new stuff'.)
By convention, we define $Y_0 = M_0$ and $Y_1 = M_1$, so that $X_0 = (0)$ and $X_1 = (0)$.

\item
\label{EP:BasicContruction}
The map $ae_n b\mapsto ap_n b$ gives a $*$-isomorphism from $X_{n+1}=M_n e_n M_n$ to $\langle M_n , p_n\rangle=M_np_nM_n$, the Jones basic construction of $M_{n-1} \subseteq M_n$ acting on $L^2(M_n,\tr_n)$.

\item
\label{EP:OtherMarkovDef}
Under the isomorphism $X_{n+1} \cong M_n p_n M_n$, the canonical non-normalized trace $\Tr_{n+1}$ on the Jones basic construction algebra $M_np_nM_n$ satisfying $\Tr_{n+1}(ap_nb) = \tr_n(ab)$ for $a,b\in M_n$ equals $d^2 \tr_{n+1}|_{X_{n+1}}$.

\item
\label{EP:NewStuff}
If $y\in Y_{n+1}$ and $x\in X_{n}$, then $yx = 0$ in $M_{n+1}$.
Hence $E_{n+1}(Y_{n+1}) \subseteq Y_{n}$.
(``The new stuff comes only from the old new stuff" \cite{MR999799}.)

\item
\label{EP:FiniteDepth}
If $Y_n =(0)$, then $Y_{k} = (0)$ for all $k\geq n$.

\end{enumerate}
\end{prop}

\begin{proof} 

\begin{enumerate}[(A)]
\item\label{inj}

We know that $d^2E_{n+1}(ye_n) = y $, so the proposed map has a left inverse.

\item

This follows directly from \eqref{inj}

\item

For $x\in M_n$, we have $\tr_{n+1}(xe_n) = \tr_n(E_{n+1}(xe_n)) = \tr_n(x E_{n+1}(e_n)) = d^{-2} \tr_n(x)$, since $E_{n+1}(e_n) = d^{-2}$.

\item

$M_{n+1} e_n = M_n e_n$ and $e_nxe_n = E_n(x)e_n$ for all $x\in M_n$.

\item

$M_ne_nM_n$ is a 2-sided ideal of $M_{n+1}$ by the pull down condition, as discussed in \eqref{pulldowniff}

\item

% It suffices to show the map is injective, which shows it is well-defined. % This was written here, but I can't make sense of it. 
First, we check that the map $\phi:ae_nb\to ap_nb$ is injective. 
Suppose $\sum a_i p_n b_i = 0$.
Then for all $a,b\in M_n$, we have $0=p_na\left(\sum a_i p_n b_i\right) bp_n = \sum E_{n}(aa_i)E_n(b_ib)p_n$, which implies $\sum E_{n}(aa_i)E_n(b_ib) = 0$ as $M_n \ni x\mapsto xp_n \in \langle M_n, p_n\rangle$ is injective.
Hence $0 = \sum E_{n}(aa_i)E_n(b_ib)e_n = e_na\left(\sum a_i e_n b_i\right) be_n$ for all $a,b\in M_n$, which implies $\sum a_i e_n b_i = 0$, showing that $\phi$ is injective. 

		To show that $\phi$ is well-defined, we may reverse the above argument, using \eqref{EP:Injective}. 
		It is clear that $\phi$ is surjective. 

\item

For $a,b\in M_n$, we have 
$\Tr_{n+1}(ap_n b) = \tr_n(ab) = \tr_n(ba) = d^2\tr_{n+1}(bae_n) = d^2 \tr_{n+1}(ae_n b)$.

\item

Since $X_0 = (0)$ and $X_1 = (0)$ by definition, we may assume $n\geq 2$.
As in the proof of \cite[Thm.~4.6.3.vi]{MR999799}, we may assume $y$ is a central projection in $M_{n+1}$ such that $y e_{n} = 0$.
Then for all $ae_{n-1} b \in X_n$, $y ae_{n-1} b = d^2 yae_{n-1} e_{n}e_{n-1} b = d^2 ae_{n-1} ye_{n} e_{n-1} b = 0$.
The final claim follows from $z_{n}E_{n+1}(y) = E_{n+1}(z_n y)= 0$ where $z_n$ is the central support of $e_{n-1}$ in $M_n$.

\item

This follows immediately from the previous fact.

\end{enumerate}

\end{proof}


Notice that by \eqref{EP:BasicContruction}, the Bratteli diagram for the inclusion $M_{n}\subset M_{n+1}$ consists of the reflection of the Bratteli diagram for the inclusion $M_{n-1} \subset M_n$, together with possibly some new edges and vertices corresponding to simple summands of $Y_{n+1}$. By \eqref{EP:NewStuff}, the new vertices at level $n+1$ only connect to the vertices that were new at level $n$. This leads to the following definition:


\begin{defn}
The \emph{principal graph} of the Markov sequence $(M_n,\tr_n)$ with Jones projections $(e_n)$ consists of the \emph{new} vertices at every level $n$ of the Bratteli diagram, together with all the edges connecting them.

A Markov sequence is said to have \emph{finite depth} if the principal graph is finite.
\end{defn}


It follows that a Markov sequence has finite depth if and only if there is $n\in \bbN$ such that $Y_n = (0)$, as in \ref{prop:ElementaryMarkov} \eqref{EP:FiniteDepth}. Let $(M_n)$ be a Markov sequence with finite depth, and take the minimal integer $n\in \bbN$ such that $Y_n=(0)$. Now notice that for $k<n$ the Bratteli diagram of $M_k\subseteq M_{k+1}$ is the Bratteli diagram of $M_{k-1}\subseteq M_{k}$ reflected upwards along with additional edges which are part of the principal graph. Because of this and the fact that the Bratteli diagram for $M_0\subseteq M_1$ is part of the principal graph we can ``unravel" the Bratteli diagram for $M_{n}\subseteq M_{n+1}$ to obtain the principal graph for the Markov sequence $(A_n,\tr_n)$.

\begin{fact}\label{BratteliPrincipal}
	If a Markov sequence $(M_n)$ has finite depth and $n\in \bbN$ is such that $Y_n=(0)$, then for $k\geq n$, there is a canonical graph isomorphism between the principal graph of $(M_n)$ and the Bratteli diagram for $M_{k}\subset M_{k+1}$.
\end{fact}

\begin{ex}
The Temperley-Lieb algebras of modulus $d\geq 2$ with the usual Jones projections form a Markov sequence. Their principal graph is $A_{\infty}$.
\end{ex}

\subsection{The Canonical Planar \texorpdfstring{$\ast$}{*}-Algebra of a Strongly Markov Inclusion}
Let $M_0\subseteq (M_1,\tr_1)$ be a strongly Markov inclusion of finite dimensional von Neumann algebras with $[M_1:M_0]=d^2$. By iterating the basic construction on this inclusion, we obtain a (strongly) Markov tower $(M_n,\tr_n)$. We will construct a planar algebra called the \textit{canonical planar $\ast$-algebra of a strongly Markov inclusion} corresponding to $M_0\subseteq (M_1,\tr_1)$. We refer the reader to \cite{MR2812459} for more details on this construction. We will need the following proposition:

\begin{prop} \label{iso}
For any $n$ the map:
\begin{align*}
\theta_n: \bigotimes_{M_0}^{n}M_1 &\rightarrow M_n\\
\text{defined by: } x_1\otimes\cdots x_n &\mapsto x_1v_1x_2v_2\cdots x_nv_n\\
\text{where } v_k&:=d^ne_ne_{n-1}\cdots e_1
\end{align*}
\end{prop}



\section{Construction of the Planar Algebra Associated to a Module Category}
\subsection{Construction of a Markov Sequence}
Given a finite depth subfactor planar algebra $Q_{\bullet}$ and a module $\cM$ over $Q_{\bullet}$, let $\cC$ and $\widetilde{\cC}$ be the multifusion categories associated to $\cG$ and $\cM$, as described in section \ref{cattheory}. Let $1_{\widetilde{\cC}}= 1_0 \oplus 1_1 \oplus 1_2$ be the tensor unit in $\widetilde{\cC}$, indexed so that $\cC$ is the non-unital $2\times 2$ unitary multifusion subcategory of $\widetilde{\cC}$ with $1_\cC = 1_0\oplus 1_1$, and $\cM\cong\cC_{20}\oplus \cC_{21}$. Overall,
$$
\cM 
=
\begin{pmatrix}
\cC_{20} & \cC_{21} 
\end{pmatrix}
\qquad\qquad
\cC
=
\begin{pmatrix}
\cC_{00} & \cC_{01} 
\\
\cC_{10} & \cC_{11}
\end{pmatrix}
\subset
\begin{pmatrix}
\cC_{00} & \cC_{01} & \cC_{02}
\\
\cC_{10} & \cC_{11} & \cC_{12}
\\
\cC_{20} & \cC_{21} & \cC_{22}
\end{pmatrix}
=
\widetilde{\cC}
$$
Let $x\in \cC_{01}$ be the strand pictured below:
$$x=
\begin{tikzpicture}[baseline=-.1cm]
	\fill[white] (0,-.7) rectangle (-.7,-.7);
	\fill[shaded] (0,-.7) rectangle (.7,.7);
	\draw (0,-.7) -- (0,.7);

\end{tikzpicture}
\ \,\,\,\,\, \text{and,}\,\,\,\,\, \overline{x}=
\begin{tikzpicture}[baseline=-.1cm]
	\fill[white] (0,-.7) rectangle (.7,.7);
	\fill[shaded] (0,-.7) rectangle (-.7,.7);
	\draw[fill] (0,-.7) -- (0,.7);

\end{tikzpicture}
\,\,\,\,\, \text{so} \,\,\,\,\, x\otimes \overline{x} = 
\begin{tikzpicture}[baseline=-.1cm]
	\fill[white] (0,-.7) rectangle (-.7,.7);
	\fill[shaded] (0,-.7) rectangle (.7,.7);
	\draw (0,-.7) -- (0,.7);
	\draw (.7,-.7) -- (.7,.7);

\end{tikzpicture}
$$
By construction, $x$ is a generating object for $\cC$. If we pick a simple object $m\in \cC_{20}$, since $\cM$ is indecomposable, we have that any object of $\cM$ is isomorphic to a direct summand of $m\otimes \xalt$ for some nonnegative integer $n$, where $$
x^{\text{alt}\otimes n}:=\underbrace{x\otimes \overline{x} \otimes x \otimes \cdots \otimes x^?}_{n \text{ tensorands}}\,,
$$ and $x^? = \overline x$ if $n$ is even and $x$ if $n$ is odd. We define $\xbaralt$ similarly.

We will construct a finite depth Markov sequence of algebras from the action of $\cC$ on $\cM$ as follows. Set $A_n=\End_{\widetilde{\cC}}(m\otimes \xalt)$. We can represent morphisms in $A_n$ as 
$$
\begin{tikzpicture}[baseline=-.1cm]
	\fill[ctwoshading] (-.15,-.7) rectangle (-.5,.7);
	\draw[thick, red] (-.15,-.7) -- (-.15,.7);
	\draw (.15,-.7) -- (.15,.7);
	\roundNbox{unshaded}{(0,0)}{.3}{0}{0}{$f$}
	\node at (.3,.5) {\scriptsize{$n$}};
	\node at (.3,-.5) {\scriptsize{$n$}};
\end{tikzpicture}
$$
where the red strand represents $m$ and the $n$ represents $x^{\text{alt}\otimes n}$. We have a faithful tracial state $\tr_n:A_n\to \C$ given by
\begin{equation}\label{eq:TraceAn}
\tr_n(f) 
\,:=\, 
\frac{1}{\dim_{\widetilde{\cC}}(m)\dim_{\widetilde{\cC}}(x)^n}\cdot\tr_{\widetilde{\cC}}(f)
\,=\,
\frac{1}{\dim_{\widetilde{\cC}}(m)}\cdot
\frac{1}{d^n}
\cdot
\left(
\begin{tikzpicture}[baseline=-.1cm, rotate=90]
	\filldraw[thick, red, ctwoshading] (-.3,.15) arc (270:90:.2cm) -- (.3,.55) arc (90:-90:.2cm);
	\draw (-.3,-.15) arc (90:270:.2cm) -- (.3,-.55) arc (-90:90:.2cm);
	\roundNbox{unshaded}{(0,0)}{.3}{0}{0}{$f$}
%	\node at (-.7,-.35) {\scriptsize{$n$}};
	\node at (0,-.7) {\scriptsize{$\overline{n}$}};
	\node at (0,.75) {\scriptsize{$\overline{m}$}};
%	\node at (.7,.35) {\scriptsize{$m$}};
\end{tikzpicture}
\right)
\end{equation}
Where $n$ and $\overline{n}$ represent $\xalt$ and $\xbaralt$, respectively. This makes each $A_n$ a finite dimensional von Neumann algebra.

\begin{rem}
	The trace can be identified with a complex scalar because $\overline{m}\otimes m\in\cC_{00}$, the red cup $1\to\overline{m}\otimes m$ and cap $\overline{m}\otimes m\to 1$ factor through the simple summand $1_0$. Since $1_0\otimes 1\cong 1_0$, the trace is a member of the $1$-dimensional algebra $\End_{\widetilde{\cC}}(1_0)\subseteq\End_{\widetilde{\cC}}(1)$. 
\end{rem}

We also have a natural tracial inclusion $A_n \rightarrow A_{n+1}$ % such that the trace restricts ($tr_{n+1}|_{A_n}=tr_n$):
\begin{equation}\label{eq:Inclusion}
\begin{tikzpicture}[baseline=-.1cm]
	\fill[ctwoshading] (-.15,-.7) rectangle (-.5,.7);
	\draw[thick, red] (-.15,-.7) -- (-.15,.7);
	\draw (.15,-.7) -- (.15,.7);
	\roundNbox{unshaded}{(0,0)}{.3}{0}{0}{$f$}
	\node at (.3,.5) {\scriptsize{$n$}};
	\node at (.3,-.5) {\scriptsize{$n$}};
\end{tikzpicture}
\mapsto
\begin{tikzpicture}[baseline=-.1cm]
	\fill[ctwoshading] (-.15,-.7) rectangle (-.5,.7);
	\draw[thick, red] (-.15,-.7) -- (-.15,.7);
	\draw (.15,-.7) -- (.15,.7);
	\draw (.45,-.7) -- (.45,.7);
	\roundNbox{unshaded}{(0,0)}{.3}{0}{0}{$f$}
	\node at (.15,.9) {\scriptsize{$n$}};
	\node at (.15,-.9) {\scriptsize{$n$}};
	\node at (.6,0) {\scriptsize{$1$}};
\end{tikzpicture}
\end{equation}
and a trace preserving conditional expectation $E_n:A_n\rightarrow A_{n-1}$, 
\begin{equation}\label{eq:ConditionalExpectationAn}
\begin{tikzpicture}[baseline=-.1cm]
	\fill[ctwoshading] (-.15,-.7) rectangle (-.5,.7);
	\draw[thick, red] (-.15,-.7) -- (-.15,.7);
	\draw (.15,-.7) -- (.15,.7);
	\roundNbox{unshaded}{(0,0)}{.3}{0}{0}{$f$}
	\node at (.3,.5) {\scriptsize{$n$}};
	\node at (.3,-.5) {\scriptsize{$n$}};
\end{tikzpicture}
\mapsto\,
\frac{1}{d}
\cdot
\left(\,\,
\begin{tikzpicture}[baseline=-.1cm]
	\fill[ctwoshading] (-.15,-.7) rectangle (-.5,.7);
	\draw[thick, red] (-.15,-.7) -- (-.15,.7);
	\draw (0,-.7) -- (0,.7);
	\draw (.15,.3) arc (180:0:.15cm) -- (.45,-.3) arc (0:-180:.15cm);
	\roundNbox{unshaded}{(0,0)}{.3}{0}{0}{$f$}
	\node at (.4,.6) {\scriptsize{$n-1$}};
	\node at (.4,-.6) {\scriptsize{$n-1$}};
	\node at (.6,0) {\scriptsize{$1$}};
\end{tikzpicture}
\right)
\end{equation}
left inverse to the inclusion.

Here, $d:=\dim_{\widetilde{\cC}}(x)=\dim_{\widetilde{\cC}}(\overline{x})$ the value of a closed loop appearing in the diagram. % There must be a better word for this. Maybe we should also remark why these are equal? In response: Emily peter's uses the term closed circles (on page 7 of her thesis), if that's the woridng you mean, and the equality follows from sphericality, maybe we should this should probably be in the preliminary discussion of the paragroup and the constructed multifusion category
Given the previously defined inclusion and that multiplication is given by vertical stacking of diagrams, it's clear that $E_n$ is $A_{n-1}$ bilinear. Similarly, we have that $tr_n=tr_{n-1} \circ E_n$.

Finally, the Jones projections for each inclusion $A_{n}\subset A_{n+1}$ are given by
\begin{equation}\label{eq:JonesProjections}
e_n
:=
\frac1d\cdot
\left(\,\,
\begin{tikzpicture}[baseline=-.1cm]
	\fill[ctwoshading] (-.3,-.7) rectangle (-.6,.7);
	\draw[thick, red] (-.3,-.7) -- (-.3,.7);
	\draw (-.1,-.7) -- (-.1,.7);
	\draw (.1,-.7) -- (.1,-.4) arc (180:0:.15cm) -- (.4,-.7);
	\draw (.1,.7) -- (.1,.4) arc (-180:0:.15cm) -- (.4,.7);
%	\node at (-.1,-1) {\scriptsize{$n-1$}};
	\node at (.3,0) {\scriptsize{$n-1$}};
	\node at (.6,.5) {\scriptsize{$1$}};
	\node at (.6,-.5) {\scriptsize{$1$}};
\end{tikzpicture}
\right)
\in
A_{n+1}.
\end{equation}
where the $n-1$ indicates $n-1$ vertical strands with appropriate shading to the right of the red strand. The Temperley-Lieb- % strange editor enforced wordwrap from here. lint?
Jones relations follow immediately from the definition and the fact that closed loops count for a factor of $d$. Similarly, 
for any $x\in A_n$ we have $e_nxe_n=E_n(x)e_n$. Clearly, $E_{n+1}(e_n)=d^{-2}\id_{m \otimes \xalt}$. Finally, the pull 
down condition holds, as for each $x\in A_{n+1}$, we have that $d^2 E_{n+1}(xe_n)e_n=xe_n$. Equivalently, $A_{n} e_n A_n$ is a 
two-sided ideal in $A_{n+1}$. Thus, the tower algebras given by the $A_n$ is indeed a Markov sequence.
 
\begin{prop}
The Markov sequence of von Neumann Algebras $\left(A_n, tr_n\right)_{n\geq 0}$ is finite depth.
\end{prop}

\begin{proof}
	Let $Y_1,\ldots Y_{r}$, be a collection of non-isomorphic simple objects spanning $\cC_{20}$. Then when $n$ is even we have:
	\[m\otimes \xalt \cong \bigoplus\limits_{j}\left(n_jY_j\right) \]
 For some nonnegative integers $n_1,\ldots n_r$ so then we have:
	\[\End_{\widetilde{\cC}}\left(m\otimes\xalt\right)\cong \bigoplus\limits_{j}\End_{\widetilde{\cC}}\left(n_j Y^l_j\right) \]
	If there are $s$ isomorphism classes of simple objects in $\cC_{21}$ we obtain a similar result. So for any $n$ the Bratteli diagram for $A_n\subseteq A_{n+1}$ has at most $r+s$ vertices, so the principal graph must be finite. 
\end{proof}

\begin{thm}
The principal graph obtained from the Markov sequence $\left(A_n, tr_n\right)_{n\geq 0}$ is independent of the choice of $m$, and is in fact isomorphic to the fusion graph for $\cM$.
\end{thm}

\begin{proof}
We will proceed by noting that because $x$ generates $\cC$ and $\cM$ is cyclic that we can assume there is some simple from every isomorphism class in $\cM$ appearing as a direct summand of $m\otimes \xalt $. Then our result from the fact that subsequent in inclusions $A_k\subseteq A_{k+1}$ are given by tensoring $\id_x$ and $\id_{\overline{x}}$ so they are determined entirely by the fusion rules.

First, recall that for a finite depth Markov sequence, the principal graph is canonically isomorphic to the Bratteli for the inclusion $A_n\subseteq A_{n+1}$ for $n$ large enough by \ref{BratteliPrincipal}. So, it suffices to show this Bratteli diagram is isomorphic to the fusion graph for $\cM$. Furthermore if $n$ is such that $A_n\subseteq A_{n+1}\subseteq A_{n+2}$ is standard then for $k\geq n$ the Bratteli diagram for $A_{k}\subseteq A_{k+1}$ is isomorphic to the Bratteli diagram for $A_n\subseteq A_{n+1}$ so without loss of generality we may assume that $n$ is even.

Let $Y_1,\ldots Y_{r}$ be a collection of non-isomorphic simple objects in $\cC_{20}$ such that every object in $\cC_{20}$ is isomorphic to a direct sum of these elements, as in the previous proof. Let $Z_{1},\ldots Z_{s}$ be a similarly defined collection of simple objects for $\cC_{21}$. 
	Let the non-negative integers $p_{i,j}$ be defined by the following equation:

$$Y_j\otimes x=\bigoplus\limits_{i}\left(p_{i,j}Z_i\right)$$

These coefficients are exactly the coefficients for the adjacency matrix of the fusion graph for $x$ acting on $\cM$.

Recall that:
$$A_n=\End_{\widetilde{\cC}}\left(m\otimes\xalt\right)\cong \bigoplus\limits_{j}\End_{\widetilde{\cC}}\left(n_j Y_j\right)$$
if we have that 
$$m \otimes \xalt \cong \bigoplus\limits_{j}\left(n_jY_j\right)$$
Note that we may take $n$ such that all $n_j$ are strictly positive, as there is some $n$ such that $Y_1\oplus\cdots \oplus Y_{r}$ is isomorphic to a direct summand of $m\otimes \xalt$ (in fact the first $n$ where this happens is exactly when the Markov sequence achieves depth). 

The inclusion of $A_n \hookrightarrow A_{n+1}$ is given by $\phi \mapsto \phi \otimes \id_{x}$. Each $\phi \in A_n$ is uniquely defined as a direct sum 
$$\bigoplus\limits_{j}\phi_j$$
where each $\phi_j \in \End_{\widetilde{\cC}}\left(n_jY_j\right)$. Then, distributing tensor products over the sum, we may write $\phi$ as 
$$\bigoplus_{j}\left(\phi_j \otimes \id_{x}\right)$$
But then:
$$\phi_j \otimes \id_{x} \in \End_{\widetilde{\cC}}\left(n_j Y_j \otimes x\right)\cong \bigoplus\limits_{i} \End_{\widetilde{\cC}}\left( p_{i,j} n_j Z_i\right)$$
Since all $n_j\neq 0$ and the inclusion is unital, it's clear that $p_{i,j}$ gives the coefficients for the Bratteli diagram.
\end{proof}



\section{The Embedding Theorem}

\begin{claim*}
A finite depth subfactor planar algebra can be embedded into the bipartite graph planar algebra of the fusion graph obtained by right-acting the subfactor planar algebra on a cyclic module.
\end{claim*}
As was described in \cite{penneys}, one can define a canonical shaded planar $*$-algebra structure on the tower of relative commutants of the base of a strongly Markov tower of algebras. % Now that this construction is summarized in a subsection of section 2, we may want to shorten this reference to there. 
This planar algebra is isomorphic to the bipartite graph planar algebra of the Bratteli diagram of the first inclusion in the tower, by theorem 3.28 of \cite{penneys}. We have shown that the Markov sequence $(A_n,\tr_n)$ defined in the previous section is finite depth. Let $r$ be the minimal integer such that the inclusion $A_{2r} \subset A_{2r+1} \subset A_{2r+2}$ is standard, and let $B_n=A_{2r+n}$. Then the canonical planar $*$-algebra $P_\bullet$ associated to $B_0\subseteq (B_1,tr_1)$ is isomorphic to the bipartite graph planar algebra of the principle graph of the tower $\left(A_{n}\right)$; indeed, by \ref{BratteliPrincipal}, the principle graph of $(A_{n},tr_n)$ is isomorphic to the Bratteli diagram of $B_0\subseteq B_1$.
If we denote the canonical $*$-planar algebra associated to $B_0\subseteq(B_1,tr_1)$ by $P_{\bullet}$, then by construction, we have the following:
\begin{align*}
	P_{n,+} &=  A'_{2r}\cap A_{2r+n} \\
	P_{n,-}  &= A'_{2r+1}\cap A_{2r+n+1} 
\end{align*}

Consider the map $\Phi$, which places $2r$ strings to the left of elements in $Q_{n,+}$  and $2r+1$ strings to the left of elements in $Q_{n,-}$:
\[ \begin{tikzpicture}[baseline]
	\draw (0,-.7) -- (0,.7);
	\roundNbox{unshaded}{(0,0)}{.3}{0}{0}{$x$}
	\node at (0,-.9) {\tiny{$n$}};
\end{tikzpicture}
\quad
\begin{tikzpicture}[baseline]
	\clip (0.5,0.9) -- (-0.5,0.9) -- (-0.5,-0.9) -- (0.5,-0.9);
	\draw [|->,thick] (-0.3,0)--(0.3,0);
	\node at (0,0.25) {\scriptsize{$\Phi$}};
\end{tikzpicture}
\quad
\begin{tikzpicture}[baseline]
	\fill[ctwoshading] (-.3,-.7) -- (-0.3,.7) -- (-.7,0.7) -- (-.7,-0.7) -- (-.3,-0.7);
	\draw[thick,red] (-0.3,-.7) -- (-0.3,0.7);
	\draw (0.45,-.7) -- (0.45,.7);
	\draw (0,-.7) -- (0,.7);
	\roundNbox{unshaded}{(0.45,0)}{.3}{0}{0}{$x$}
	\node at (0.45,-.9) {\tiny{$n$}};
	\node at (0,-0.9){\tiny{$2r$}};
\end{tikzpicture} \]
This map sends $Q_{n , +}$ to $P_{n,+}$ and $Q_{n , -}$ to $P_{n,-}$. In order to check that $\Phi(Q_{\bullet})$ is a sub $\ast$-planar algebra, we use lemma 2.48 in \cite{penneys}.
We shall state it here for the reader's convenience:
\begin{lem}[Preservation under action of tangles]
Suppose that $P_\bullet$ is a planar $\ast$-algebra with modulus $d \neq 0$ and $Q_{n,\pm} \subset P_{n,\pm}$ are $\ast$-subalgebras which are preserved by the action of the following tangles:

\begin{enumerate}[(1)]
\item Multiplication by tangles of the form $E_n = 
\begin{tikzpicture}[baseline]
	\draw (-.1,-.7) -- (-.1,.7);
	\draw (.5,-.7) -- (.5,-.4) arc (180:0:.15cm) -- (.8,-.7);
	\draw (.5,.7) -- (.5,.4) arc (-180:0:.15cm) -- (.8,.7);
%	\node at (-.1,-1) {\scriptsize{$n-1$}};
	\node at (-.1,-.9) {\tiny{$n-1$}};
\end{tikzpicture}
$
\item 
\begin{enumerate}[(i)]
\item (Right Capping) $\alpha_n = 
\begin{tikzpicture}[baseline]
	\draw (0,-.7) -- (0,.7);
	\draw (.15,.3) arc (180:0:.15cm) -- (.45,-.3) arc (0:-180:.15cm);
	\roundNbox{unshaded}{(0,0)}{.3}{0}{0}{}
	\node at (0,-0.9) {\tiny{$n-1$}};
\end{tikzpicture}
: P_{n,+} \to P_{n-1,+}
$
\item (Inclusion) $\beta_n = 
\begin{tikzpicture}[baseline]
	\draw (0,-.7) -- (0,0.7);
	\draw (0.45,-.7) -- (0.45,.7);
	\roundNbox{unshaded}{(0,0)}{.3}{0}{0}{}
	\node at (0,-.9) {\tiny{$n$}};
	\node at (0.45,-.9) {\tiny{$1$}};
\end{tikzpicture}
: P_{n,+} \to P_{n+1,+}
$
\item (Left Capping) $\gamma^{+}_n =
\begin{tikzpicture}[baseline]
	\fill[relativecommutantshading] (0,-.7) -- (-.8,-.7) -- (-.8,.7) -- (0,.7) -- (0,-.7);
	\fill[unshaded] (-.15,.3) arc(0:180:.15cm) -- (-.45,-.3)arc(-180:0:0.15cm)--(-.15,.3);
	\draw (0,-.7) -- (0,.7);
	\draw (-.15,.3) arc (0:180:.15cm) -- (-.45,-.3) arc (-180:0:.15cm);
	\roundNbox{unshaded}{(0,0)}{.3}{0}{0}{}
	\node at (0,-0.9) {\tiny{$n-1$}};
\end{tikzpicture}
: P_{n,+} \to P_{n-1,-}
$ 
\end{enumerate}
\item (Left inclusion) $i^{-}_n =
\begin{tikzpicture}[baseline]
	\fill[relativecommutantshading] (-0.45,.7) -- (0,.7) -- (0,-.7) -- (-0.45,-.7) -- (-0.45,.7);
	\draw (0,-.7) -- (0,0.7);
	\draw (-0.45,-.7) -- (-0.45,.7);
	\roundNbox{unshaded}{(0,0)}{.3}{0}{0}{}
	\node at (0,-.9) {\tiny{$n$}};
	\node at (-0.45,-.9) {\tiny{$1$}};
\end{tikzpicture}
: P_{n,-} \to P_{n+1,+}
$ 
\end{enumerate}
Then the $Q_{n,\pm}$ are preserved under every tangle and thus form a planar $\ast$-subalgebra $Q_\bullet \subset P_\bullet$
\end{lem}
We denote traces in $Q_{\bullet}$ by $\overline{\tr}$ and traces in $P_{\bullet}$ by $\tr$. We denote the Jones projections in $Q_{\bullet}$ by $E_{j}$ and the Jones projections in  $P_{\bullet}$ by $F_{j}$.


\begin{thm}
The image of the map $\Phi:Q_{\bullet} \to P_{\bullet}$ is a $\ast$-planar subalgebra of $P_{\bullet}$.
\end{thm} 

\begin{proof}
Observe that for all $x,y \in Q_{\bullet}$, we have that 
\begin{align}
	\Phi(x^{*}) &= \Phi(x)^{\ast} \\
	\Phi(xy) &= \Phi(x)\Phi(y) \\
	\overline{\tr}_{n}(x) &= \frac{1}{\dim_\cC(x)^{2r}}\cdot\tr_{n}(\Phi(x)) \label{eq3}
\end{align}

	To prove the injectivity of $\Phi$, we can use the faithfulness of the traces and equation \eqref{eq3}.
	In section \todo{Cover how the relative commutant algebra looks} we see that in the relative commutant planar algebra, the tangles described in Lemma 2.48 are associated to known functions. Thus we can check that $\Phi(Q_\bullet)$ is preserved under the specified tangles by using both algebraic and pictorial relations. We now proceed to check the requirements of Lemma 2.48:
\begin{enumerate}[(1)]
\item By the construction of $P_{\bullet}$, we have that $\Phi(E_j)=F_j$. Thus, $\Phi(Q_{n,\pm})$ is closed under multiplication by Jones projections. 
\item \begin{enumerate}[(i)]
\item \label{condex} Clearly: $\Phi(\alpha_n(x)) = \alpha_{2r+n}(\Phi(x))$.
\item Similarly to \eqref{condex}, it is clear that $\Phi(\beta_n(x)) = \beta_{2r+n}(\Phi(x))$. In fact, the inclusion from $P_{n,+} \to P_{n+1,+}$ is just the restriction of the inclusion from $Q_{2r+n,+} \to Q_{2r+n+1,+}$. 
\item In the section 2, %cite whatever lemma this will be
 we showed that the left capping tangle $\gamma_n$ in the canonical $*$-planar algebra is given by:
\[
 \gamma^{+}_n(x) = \frac{1}{d^2}\sum_{b \in B}bxb^\ast
\]
where $B$ is a Pimnser Popa basis of $B_1$ over $B_0$. We will now check that $\gamma^{+}_n$ preserves $\Phi(Q_{\bullet})$. For any $x \in Q_{n,+}$, we have 

\centerline{
  \begin{minipage}{\linewidth}
			\[
				d^2\gamma^+_n(\Phi(x))
				=
\sum_{b\in B}
\begin{tikzpicture}[baseline]
	\fill[ctwoshading] (-1,-2.1) -- (-1,2.1) -- (-.6,2.1) -- (-.6,-2.1) -- (-1,-2.1);
	\fill[shaded] (0,2.1) -- (0,0.7) -- (0.3,0.7) -- (0.3,-0.7) -- (0,-0.7) -- (0,-2.1) -- (0.6,-2.1) -- (0.6,2.1) -- (0,2.1);
	\draw[thick,red] (-0.6,-2.1) -- (-0.6,2.1);
	\draw (0.6,-2.1) -- (0.6,2.1);
	\draw (0.3,-.7) -- (0.3,.7);
	\draw (0,-2.1) -- (0,2.1);
	\roundNbox{unshaded}{(0.45,0)}{.3}{0}{0}{$x$}
	\roundNbox{unshaded}{(0.15,1)}{.3}{0.6}{0}{$b$}
	\roundNbox{unshaded}{(0.15,-1)}{.3}{0.6}{0}{$b^\ast$}
	\node at (0.9,-2.3) {\tiny{$n-1$}};
	\node at (0,2.2) {\tiny{$2r+1$}};
	\node at (-.15,0) {\tiny{$2r$}};
\end{tikzpicture} 
=
\sum_{b\in B}
\begin{tikzpicture}[baseline]
	\fill[ctwoshading] (-1,-2.1) -- (-1,2.1) -- (-.6,2.1) -- (-.6,-2.1) -- (-1,-2.1);
	\fill[shaded] (0,2.1) -- (0,0.7) -- (0.3,0.7) -- (0.3,0.25)arc (-180:0:.15cm)-- (0.6,1.8)arc(180:0:.3cm) -- (1.2,-1.8) arc(0:-180:0.3cm) -- (0.6,-0.25) arc(0:180:0.15cm) -- (0.3,-0.7) -- (0,-.7) -- (0,-2.1) -- (1.5,-2.1) -- (1.5,2.1);
	\draw[thick,red] (-0.6,-2.1) -- (-0.6,2.1);
	\draw (0.3,.7)--(0.3,0.25) arc (-180:0:.15cm)-- (0.6,1.8) arc(180:0:.3cm) -- (1.2,.15);
	\draw (0.3,-.7)--(0.3,-0.25) arc (180:0:.15cm)-- (0.6,-1.8) arc(-180:0:.3cm) -- (1.2,-.15);
	\draw (0,-2.1) -- (0,2.1);
	\draw (1.5,-2.1) -- (1.5,2.1);
	\draw[dashed] (-1,1.5) --  (-1,-1.5) --  (0.8,-1.5) --  (0.8,1.5) -- (-1,1.5);
	\roundNbox{unshaded}{(1.35,0)}{.3}{0}{0}{$x$}
	\roundNbox{unshaded}{(0.15,1)}{.3}{0.6}{0}{$b$}
	\roundNbox{unshaded}{(0.15,-1)}{.3}{0.6}{0}{$b^\ast$}
	\node at (1.8,-2.3) {\tiny{$n-1$}};
	\node at (0,2.2) {\tiny{$2r+1$}};
	\node at (-.15,0) {\tiny{$2r$}};
\end{tikzpicture} 
=
\begin{tikzpicture}[baseline]
	\fill[ctwoshading] (-1,-2.1) -- (-1,2.1) -- (-.6,2.1) -- (-.6,-2.1) -- (-1,-2.1);
	\fill[shaded] (0,2.1) -- (1.5,2.1) -- (1.5,-2.1) -- (0,-2.1) -- (0,2.1);
	\fill[unshaded] (0.6,1.8) arc(180:0:.3cm) -- (1.2,-1.8)arc(0:-180:0.3cm) -- (0.6,1.8); 
	\draw[thick,red] (-0.6,-2.1) -- (-0.6,2.1);
	\draw (0.6,1.8) arc(180:0:.3cm) -- (1.2,.15);
	\draw (0.6,-1.8) arc(-180:0:.3cm) -- (1.2,-.15);
	\draw (0.6,1.8) -- (0.6,-1.8);
	\draw (0,-2.1) -- (0,2.1);
	\draw (1.5,-2.1) -- (1.5,2.1);
	\draw[dashed] (-1,1.5) --  (-1,-1.5) --  (0.8,-1.5) --  (0.8,1.5) -- (-1,1.5);
	\roundNbox{unshaded}{(1.35,0)}{.3}{0}{0}{$x$}
	\node at (1.8,-2.3) {\tiny{$n-1$}};
	\node at (0,2.2) {\tiny{$2r+1$}};
	\node at (0.5,0) {\tiny{$1$}};
\end{tikzpicture} 
=
\begin{tikzpicture}[baseline]
	\fill[ctwoshading] (-1,-2.1) -- (-1,2.1) -- (-.6,2.1) -- (-.6,-2.1) -- (-1,-2.1);
	\fill[shaded] (0,2.1) -- (.9,2.1) -- (.9,-2.1) -- (0,-2.1) -- (0,2.1);
	\fill[unshaded] (0.3,0.7) arc(180:0:.15cm) -- (0.6,-0.7)arc(0:-180:0.15cm) -- (0.3,0.7); 
	\draw[thick,red] (-0.6,-2.1) -- (-0.6,2.1);
	\draw (0.3,0.7) arc(180:0:.15cm) -- (0.6,.15);
	\draw (0.3,-0.7) arc(-180:0:.15cm) -- (0.6,-.15);
	\draw (0.3,0.7) -- (0.3,-0.7);
	\draw (0,-2.1) -- (0,2.1);
	\draw (0.9,-2.1) -- (0.9,2.1);
	\roundNbox{unshaded}{(0.75,0)}{.3}{0}{0}{$x$}
	\node at (1.2,-2.3) {\tiny{$n-1$}};
	\node at (0,2.2) {\tiny{$2r+1$}};
\end{tikzpicture}
	\]
  \end{minipage}
}
Clearly, the last diagram depicts an element of  $\Phi(Q_{\bullet})$. In fact, we have $\gamma^+_n(\Phi(x)) =\Phi(\gamma^+_n(x))$.
\end{enumerate}
\item The negative inclusion $i^{-}_{n} : P_{n,-} \to P_{n+1,+}$ is just the identity on the relative commutant planar algebra. Let $\overline{i^{-}_{n}}$ be the negative inclusion in $Q_{n,\pm}$. Graphically, this is equivalent to adding a string on the left. Thus, for $x$ in $Q_{n,-}$ we have that:
\[
\begin{tikzpicture}[baseline]
	\fill[ctwoshading] (-.9,-.7) -- (-0.9,.7) -- (-1.3,0.7) -- (-1.3,-0.7) -- (-0.9,-0.7);
	\fill[shaded] (-0.45,-0.7) -- (-0.45,0.7) -- (0.15,0.7) -- (0.15,-0.7) -- (-0.45,-0.7);
	\draw[red, thick] (-0.9,-0.7) -- (-0.9,0.7);
	\draw (-0.45,-0.7) -- (-0.45,0.7);
	\draw (0.15,-0.7) -- (0.15,0.7);
	\roundNbox{unshaded}{(0.15,0)}{.3}{0}{0}{$x$};
	\node at (0.15,0.8){\tiny{$n$}};
	\node at (-0.45, -0.9){\tiny{$2r+1$}};
	\node at (-0.30,-1.6){\scriptsize{$i_n(\Phi(x))$}};
\end{tikzpicture}
\quad
=
\quad
\begin{tikzpicture}[baseline]
	\fill[ctwoshading] (-.9,-.7) -- (-0.9,.7) -- (-1.3,0.7) -- (-1.3,-0.7) -- (-0.9,-0.7);
	\fill[shaded] (-0.45,-0.7) -- (-0.45,0.7) -- (0.15,0.7) -- (0.15,-0.7) -- (-0.45,-0.7);
	\draw[red, thick] (-0.9,-0.7) -- (-0.9,0.7);
	\draw (-0.45,-0.7) -- (-0.45,0.7);
	\draw (0.15,-0.7) -- (0.15,0.7);
	\roundNbox{unshaded}{(0.15,0)}{.3}{0}{0}{$x$};
	\node at (0.15,0.8){\tiny{$n$}};
	\node at (-0.45, -0.9){\tiny{$2r+1$}};
	\node at (-0.30,-1.6){\scriptsize{$\Phi(x)$}};
\end{tikzpicture}
\quad
=
\quad
\begin{tikzpicture}[baseline]
	\fill[ctwoshading] (-.9,-.7) -- (-0.9,.7) -- (-1.3,0.7) -- (-1.3,-0.7) -- (-0.9,-0.7);
	\fill[shaded] (0.15,-0.7) -- (0.15,0.7) -- (0.6,0.7) -- (0.6,-0.7) -- (0.15,-0.7);
	\draw[red, thick] (-0.9,-0.7) -- (-0.9,0.7);
	\draw (-0.45,-0.7) -- (-0.45,0.7);
	\draw (0.6,-0.7) -- (0.6,0.7);
	\draw (0.15,-0.7) -- (0.15,0.7);
	\roundNbox{unshaded}{(.6,0)}{0.3}{0}{0}{$x$};
	\draw[dashed] (-0.15,-1.1) -- (-0.15,1.1) -- (1.05,1.1) --  (1.05,-1.1) -- (-0.15,-1.1);
	\node at (0.6,0.8){\tiny{$n$}};
	\node at (-0.45, -0.9){\tiny{$2r$}};
	\node at (0.15,-0.9){\tiny{1}};
	\node at (0.15,-1.6){\scriptsize{$\Phi(\overline{i^{-}_{n}}(x))$}};
\end{tikzpicture}\]
\end{enumerate}
\end{proof}

We have checked that $\Phi(Q_{\bullet}) \subset P_{\bullet}$ is a $\ast$-planar algebra inclusion. Let $G_\bullet$ be the bipartite graph planar algebra of the fusion graph of $x$ acting on $\cM$. Then by Theorem 3.33 in \cite{penneys}
we have that $P_\bullet$ is $\ast$-planar algebra isomorphic to $G_\bullet$. Thus, we have an embedding $Q_\bullet$ into $G_\bullet$.

\begin{cor}[The Embedding Theorem]
	A subfactor planar algebra $Q_\bullet$ can be embedded into the bipartite graph planar algebra of the principle graph of a cyclic $Q_\bullet$-module.
\end{cor}



\bibliographystyle{amsalpha}
{\footnotesize{
\bibliography{embedding_theorem}
}}
\end{document}
