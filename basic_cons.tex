\documentclass[a4paper,10pt]{report}
\usepackage[utf8]{inputenc}
\usepackage{amsmath}
\usepackage{amssymb}
\usepackage{fancyhdr}
\usepackage{enumitem}
\usepackage[margin=1in]{geometry}
\usepackage{mathrsfs}
\usepackage{bm}
\usepackage{tikz-cd}

\newcommand{\set}[1]{\{#1\}}
\newcommand{\ggen}[1]{\langle#1\rangle}
\newcommand{\rgen}[1]{\left(#1\right)}
\renewcommand{\mod}{\text{ mod }}
\newcommand{\ol}[1]{\overline{#1}}

\renewcommand{\O}{\operatorname{O}} % Bound otherwise
\renewcommand{\o}{\operatorname{o}}
\newcommand{\T}{\text{yes}}
\newcommand{\F}{\text{no}}

\newcommand{\N}{\mathbb{N}}
\newcommand{\Q}{\mathbb{Q}}
\newcommand{\Z}{\mathbb{Z}}
\newcommand{\R}{\mathbb{R}}
\newcommand{\C}{\mathbb{C}}
\newcommand{\tens}{\otimes}
\newcommand{\idl}{\mathfrak}
\newcommand{\cat}[1]{\rgen{\textbf{#1}}}
\newcommand{\op}{{op}}
\newcommand{\OP}{\text{OP}}

\DeclareMathOperator{\Dic}{Dic}
% \DeclareMathOperator{\deg}{deg}
\DeclareMathOperator{\Sym}{Sym}
\DeclareMathOperator{\stab}{stab}
\DeclareMathOperator{\D}{D}
\DeclareMathOperator{\Aut}{Aut}
\DeclareMathOperator{\Inn}{Inn}
\DeclareMathOperator{\Cl}{Cl}
\DeclareMathOperator{\End}{End}
% \DeclareMathOperator{\gcd}{gcd}
\DeclareMathOperator{\lcm}{lcm}
\DeclareMathOperator{\orb}{orb}
\DeclareMathOperator{\Hom}{Hom}
\DeclareMathOperator{\Nat}{Nat}
\DeclareMathOperator{\Fun}{Fun}
\DeclareMathOperator{\img}{img}
\DeclareMathOperator{\GL}{GL}
\DeclareMathOperator{\vspan}{span}
\DeclareMathOperator{\tr}{tr}
\DeclareMathOperator{\conj}{conj}
\DeclareMathOperator{\Ind}{Ind}
\DeclareMathOperator{\Res}{Res}
\DeclareMathOperator{\supp}{supp}
\DeclareMathOperator{\Ann}{Ann}
\DeclareMathOperator{\obj}{obj}
\DeclareMathOperator{\coker}{coker}
\DeclareMathOperator{\dom}{dom}
\DeclareMathOperator{\Tor}{Tor}
\DeclareMathOperator{\Ext}{Ext}
\DeclareMathOperator{\Gal}{Gal}
\DeclareMathOperator{\fchar}{char}
\DeclareMathOperator{\id}{id}

\newcommand{\reg}{{reg}}
\newcommand{\rar}[1]{\xrightarrow{#1}}
\newcommand{\pdot}{\bullet}
\newcommand{\del}{\partial}
% \newcommand{\modz}{(Mod-\Z)}
\newcommand{\colim}{\varinjlim}
\newcommand{\cocolim}{\varprojlim}

\begin{document}
% \maketitle

\pagestyle{fancy}	
\fancyhf{} % Reset headers and footers
\lhead{Peter Huston\\
\today}
\setlength{\headheight}{60pt} 

\begin{center}
	\textbf{The Finite Dimensional Basic Construction}
\end{center}

Suppose $A\subseteq B$ is a unital inclusion of finite dimensional Von-Neumann algebras, denoted by $\iota:A\to B$. We know that $A$ is of the form $\oplus_{i=1}^nA_i$, where $A_i\cong M_{\alpha_i}(\C)$, and $B$ is of the form $\oplus_{j=1}^mB_j$, where $B_j\cong M_{\beta_j}(\C)$. Let $p_i$ be the central projection in $A$ with $p_iA=A_i$, and let $q_j$ be the central projection in $B$ with $q_jB=B_j$. From our study of unital inclusions of multimatrix algebras, we know that $p_iB_jp_i\cong A_i\otimes M_{\Lambda_{ij}}(\C)$. In particular, we can find a system $(r_{i,j}^{a,b})_{a,b=1}^{\Lambda_i,j}$ of (rank $\alpha_i$) matrix units in the compression of $B$ by $p_iq_j$ so that for each $a$ from $1$ to $\Lambda_{ij}$, we have $r_{i,j}^{a,a}B_jr_{i,j}^{a,a}\cong A_i$, and if $\mu:A_i\otimes\C[r_{i,j}^{a,b}]_{a,b}\to B$ is induced by the multiplication of $B$, then the triangle 
\[\begin{tikzcd}
	A_i\ar[r,"\id_{A_i}\otimes1_{\C}"]\ar[dr,"\iota"] & A_i\otimes\C[r_{i,j}^{a,b}]_{a,b}\ar[d,"\mu"]\\
	& B
\end{tikzcd}\]
commutes. Since this situation arises so frequently in what follows, if $X$ is an algebra and $Y$ and $Z$ are subspaces (usually algebras) of $X$, we permit $Y\tens Z$ to denote the subalgebra $YZ$ of $X$ provided that the multiplication $\mu:Y\otimes Z\to X$ is an injection. This may be considered a tensor product internal to $X$, in the same sense as we often discuss direct sums internal to some algebra. 

\end{document}
